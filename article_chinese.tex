\documentclass[]{article}
\usepackage[T1]{fontenc}
\usepackage{lmodern}
\usepackage{amssymb,amsmath}
\usepackage{ifxetex,ifluatex}
\usepackage{fixltx2e} % provides \textsubscript
% use upquote if available, for straight quotes in verbatim environments
\IfFileExists{upquote.sty}{\usepackage{upquote}}{}
\ifnum 0\ifxetex 1\fi\ifluatex 1\fi=0 % if pdftex
  \usepackage[utf8]{inputenc}
\else % if luatex or xelatex
  \ifxetex
    \usepackage{mathspec}
    \usepackage{xltxtra,xunicode}
  \else
    \usepackage{fontspec}
  \fi
  \defaultfontfeatures{Mapping=tex-text,Scale=MatchLowercase}
  \newcommand{\euro}{€}
\fi
% use microtype if available
\IfFileExists{microtype.sty}{\usepackage{microtype}}{}
\usepackage[margin=1in]{geometry}
\ifxetex
  \usepackage[setpagesize=false, % page size defined by xetex
              unicode=false, % unicode breaks when used with xetex
              xetex]{hyperref}
\else
  \usepackage[unicode=true]{hyperref}
\fi
\hypersetup{breaklinks=true,
            bookmarks=true,
            pdfauthor={段永瑞},
            pdftitle={考虑DEA效率的上市公司财务困境预测},
            colorlinks=true,
            citecolor=blue,
            urlcolor=blue,
            linkcolor=magenta,
            pdfborder={0 0 0}}
\urlstyle{same}  % don't use monospace font for urls
\setlength{\parindent}{0pt}
\setlength{\parskip}{6pt plus 2pt minus 1pt}
\setlength{\emergencystretch}{3em}  % prevent overfull lines
\setcounter{secnumdepth}{5}

\title{考虑DEA效率的上市公司财务困境预测}
\author{段永瑞}
\date{09/26/2014}
\usepackage{ctex}

\begin{document}

\begin{center}
\huge 考虑DEA效率的上市公司财务困境预测 \\[0.2cm]
\end{center}
\begin{center}
\large \emph{段永瑞}\\[0.1cm]
\end{center}
\begin{center}
\large \emph{09/26/2014} \\
\end{center}
\normalsize


\subsection{引言}

\subsection{文献综述}

\subsubsection{数据包络分析}

1978年Charnes, Cooper 和 Rhodes
提出了数据包络分析方法(DEA)。由于该方法具有很多优势,例如,评价结果与量纲无关;其研究对象可以是多投入多产出的过程;不要求明确的生产函数等,DEA在各领域得到了广泛的应用。最初提出的是CCR模型,该模型要求投入产出均为正数,但是在现实生活中,产出可能是负数,例如利润。本文介绍一种修改过的SBM来解决此问题(Tone
2004,Düzakin,E.,Düzakin,Hatice,2007)。 SBM是Tone
2001提出的一种基于松弛变量来量度效率的方法,模型如下:

\[
    \begin{aligned}
    \min & \rho=\frac{1-(1/m)\sum_{i=1}^{m}s_{i}^{-}/x_{io}}{1+(1/s)\sum_{r=1}^{s}s_{r}^{+}/y_{ro}}\\
    \textrm{s.t.} & \mathbf{x_{o}}=\mathbf{X\lambda}+\mathbf{s^{-}}\\
     & \mathbf{y_{o}}=\mathbf{Y\lambda}+\mathbf{s^{+}}\\
     & \mathbf{\lambda}\geq\mathbf{0},\mathbf{s^{-}}\geq\mathbf{0},\mathbf{s^{+}}\geq\mathbf{0}
    \end{aligned}
\]

同样的,该模型不能处理负产出。(Tone
2004,Düzakin,E.,Düzakin,Hatice,2007)提出一个解决方法,假设$y_{ro}<0$,我们可以做如下变换:

\[
\begin{aligned}
\overline{y}_{r}^{+} & =\textrm{Max}_{j=1,2,...n.}\{y_{rj}|y_{rj}>0\}\\
\underline{y}_{r}^{-} & =\textrm{Min}_{j=1,2,...n.}\{y_{rj}|y_{rj}>0\}
\end{aligned}
\]

我们用如下方法替换目标函数中的 $s_r^+/y_{ro}$ 项:如果 $y_{ro}<0$ ,且
$\overline{y}_{r}^{+}>\underline{y}_{r}^{+}$,那么该项被替换为

\[
s_r^{+}/\frac{\underline{y}_{r}^{+}(\overline{y}_{r}^{+}-\underline{y}_{r}^{+})}{\overline{y}_{r}^{+}-y_{ro}}
\]

注意我们没有替换约束中的$y_{ro}$。

经过这样的处理,可以使得所有的负产出均严格小于原最小正产出$\underline{y}_{r}^{+}$。且产出越负,其转换后的正值越小。
\#\#\# 随机森林

\subsubsection{支持向量机}

\subsubsection{朴素贝叶斯}

\subsection{结果讨论}

\subsubsection{变量选择}

\paragraph{DEA效率变量选择}\label{dea}

\paragraph{预测数据变量选择}

\subsubsection{}\label{section}

\end{document}
