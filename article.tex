\documentclass[]{article}
\usepackage[T1]{fontenc}
\usepackage{lmodern}
\usepackage{amssymb,amsmath}
\usepackage{ifxetex,ifluatex}
\usepackage{fixltx2e} % provides \textsubscript
% use upquote if available, for straight quotes in verbatim environments
\IfFileExists{upquote.sty}{\usepackage{upquote}}{}
\ifnum 0\ifxetex 1\fi\ifluatex 1\fi=0 % if pdftex
  \usepackage[utf8]{inputenc}
\else % if luatex or xelatex
  \ifxetex
    \usepackage{mathspec}
    \usepackage{xltxtra,xunicode}
  \else
    \usepackage{fontspec}
  \fi
  \defaultfontfeatures{Mapping=tex-text,Scale=MatchLowercase}
  \newcommand{\euro}{€}
\fi
% use microtype if available
\IfFileExists{microtype.sty}{\usepackage{microtype}}{}
\usepackage[margin=1in]{geometry}
\usepackage{color}
\usepackage{fancyvrb}
\newcommand{\VerbBar}{|}
\newcommand{\VERB}{\Verb[commandchars=\\\{\}]}
\DefineVerbatimEnvironment{Highlighting}{Verbatim}{commandchars=\\\{\}}
% Add ',fontsize=\small' for more characters per line
\usepackage{framed}
\definecolor{shadecolor}{RGB}{248,248,248}
\newenvironment{Shaded}{\begin{snugshade}}{\end{snugshade}}
\newcommand{\KeywordTok}[1]{\textcolor[rgb]{0.13,0.29,0.53}{\textbf{{#1}}}}
\newcommand{\DataTypeTok}[1]{\textcolor[rgb]{0.13,0.29,0.53}{{#1}}}
\newcommand{\DecValTok}[1]{\textcolor[rgb]{0.00,0.00,0.81}{{#1}}}
\newcommand{\BaseNTok}[1]{\textcolor[rgb]{0.00,0.00,0.81}{{#1}}}
\newcommand{\FloatTok}[1]{\textcolor[rgb]{0.00,0.00,0.81}{{#1}}}
\newcommand{\CharTok}[1]{\textcolor[rgb]{0.31,0.60,0.02}{{#1}}}
\newcommand{\StringTok}[1]{\textcolor[rgb]{0.31,0.60,0.02}{{#1}}}
\newcommand{\CommentTok}[1]{\textcolor[rgb]{0.56,0.35,0.01}{\textit{{#1}}}}
\newcommand{\OtherTok}[1]{\textcolor[rgb]{0.56,0.35,0.01}{{#1}}}
\newcommand{\AlertTok}[1]{\textcolor[rgb]{0.94,0.16,0.16}{{#1}}}
\newcommand{\FunctionTok}[1]{\textcolor[rgb]{0.00,0.00,0.00}{{#1}}}
\newcommand{\RegionMarkerTok}[1]{{#1}}
\newcommand{\ErrorTok}[1]{\textbf{{#1}}}
\newcommand{\NormalTok}[1]{{#1}}
\ifxetex
  \usepackage[setpagesize=false, % page size defined by xetex
              unicode=false, % unicode breaks when used with xetex
              xetex]{hyperref}
\else
  \usepackage[unicode=true]{hyperref}
\fi
\hypersetup{breaklinks=true,
            bookmarks=true,
            pdfauthor={Yongrui Duan, Fayou Zhang},
            pdftitle={Hybrid of DEA and Randomforest to predict shangshigongsi cai wu kun jing},
            colorlinks=true,
            citecolor=blue,
            urlcolor=blue,
            linkcolor=magenta,
            pdfborder={0 0 0}}
\urlstyle{same}  % don't use monospace font for urls
\setlength{\parindent}{0pt}
\setlength{\parskip}{6pt plus 2pt minus 1pt}
\setlength{\emergencystretch}{3em}  % prevent overfull lines
\setcounter{secnumdepth}{5}

\title{Hybrid of DEA and Randomforest to predict shangshigongsi cai wu kun jing}
\author{Yongrui Duan, Fayou Zhang}
\date{09/26/2014}

\begin{document}

\begin{center}
\huge Hybrid of DEA and Randomforest to predict shangshigongsi cai wu kun jing \\[0.2cm]
\end{center}
\begin{center}
\large \emph{Yongrui Duan, Fayou Zhang}\\[0.1cm]
\end{center}
\begin{center}
\large \emph{09/26/2014} \\
\end{center}
\normalsize


\subsection{Abstract}\label{abstract}

\subsection{introduction}\label{introduction}

Business failure prediction is crutial for investors, stock holders,
managers, employees and government officials, and thus has been an hot
topic in academic studies.

Many technical methods have been applied to predict business failure.
Amoung which are XX categories: XX, XX, XX, and XX. XX(author) made an
thorough review on XX.

\subsection{Literature Review}\label{literature-review}

\subsubsection{Data Envelopment
Aanaysis(DEA)}\label{data-envelopment-aanaysisdea}

DEA is a nonparametric method proposed by Charnes, Cooper and Rhodes in
1978. it has been applied to many fileds Because of its many advantages:
it does not require any assumptions to be made about the distribution of
inefficiency and it does not require a particular functional form on the
data in determining the froniter. it is capable of being used with any
input-output measurement, and capable of handling multiple inputs and
outputs. The CCR model requires inputs and outputs to be positive, which
may be not applied in real life. for example, profits as an output may
be negative. In this paper, we will introduce a modified SBM model (Tone
2004,Düzakin,E.,Düzakin,Hatice,2007) to tackle this problem. The SBM
model is as follows:

\[
    \begin{aligned}
    \min & \rho=\frac{1-(1/m)\sum_{i=1}^{m}s_{i}^{-}/x_{io}}{1+(1/s)\sum_{r=1}^{s}s_{r}^{+}/y_{ro}}\\
    \textrm{s.t.} & \mathbf{x_{o}}=\mathbf{X\lambda}+\mathbf{s^{-}}\\
     & \mathbf{y_{o}}=\mathbf{Y\lambda}+\mathbf{s^{+}}\\
     & \mathbf{\lambda}\geq\mathbf{0},\mathbf{s^{-}}\geq\mathbf{0},\mathbf{s^{+}}\geq\mathbf{0}
    \end{aligned}
\]

However, this model can not handle negative output.(Tone
2004,Düzakin,E.,Düzakin,Hatice,2007)proposed an solution to this
problem. Assuming $y_{ro}<0$, following transformations were applied:

\[
\begin{aligned}
\overline{y}_{r}^{+} & =\textrm{Max}_{j=1,2,...n.}\{y_{rj}|y_{rj}>0\}\\
\underline{y}_{r}^{-} & =\textrm{Min}_{j=1,2,...n.}\{y_{rj}|y_{rj}>0\}
\end{aligned}
\]

The term $s_r^+/y_{ro}$ in the objective function as replaced as
follows:if $y_{ro}<0$ ,and
$\overline{y}_{r}^{+}>\underline{y}_{r}^{+}$,then it will be replaced
with

\[
s_r^{+}/\frac{\underline{y}_{r}^{+}(\overline{y}_{r}^{+}-\underline{y}_{r}^{+})}{\overline{y}_{r}^{+}-y_{ro}}
\]

Note that the term $y_{ro}$ are not replaced in the constraint.

After transformation, all negaitve output were transformed to be
positive and strictly less than $\underline{y}_{r}^{+}$ . and the more
negative an output is, the less its tranformation value.

\subsubsection{Random Forest}\label{random-forest}

Random forest, developed by Leo Breiman(1) and Adele Cutler(2), is an
ensemble learning method for classification and regression that operate
by constructing a multitude of decision trees. For a given training
dataset, $A = {(X_1,y_1),(X_2,y_2),\cdots,(X_n,y_n)}$,Where
$X_i = 1,2,\cdots,n$, is a variable or vector and $y_i$ is its
corresponding property or class label; the basic RF algorithm is
presented as follows:

\paragraph{Bootstrap sample.}\label{bootstrap-sample.}

Each training set is drawn with replacement from the original dataset A.
Bootstrapping allows replacement, so that some of the samples will be
repeated in the sample, while others will be ``left out'' of the sample.
The ``left out'' samples constitute the ``Out-of bag (OOB)'' which has,
for example, one-third, of samples in A which are used later to get a
running unbiased estimate of the classification error as trees are added
to the forest and variable imp ortance

\paragraph{Growing trees.}\label{growing-trees.}

For each bootstrap sample, a tree is grown m variables $(m_{try})$ are
selected at random from all n variables $(m_{try} \leq n)$ and the best
split of all $(m_{try})$ is used at each node. Each tree is grown to the
largest extent (until no further splitting is possible) and no pruning
of the trees occurs.

\paragraph{OOB error estimate.}\label{oob-error-estimate.}

Each tree is constructed on the bootstrap sample. The OOB samples are
not used and therefore regarded as a test set to provide an unbiased
estimate of the prediction accuracy. Each OOB sample is put down the
constructed trees to get a classification. A test set classification is
formed. At the end of the run, take k to be the class which got most of
the ``votes'' every time sample n was OOB. The proportion of times that
k is not the true class of n averaged over all samples is the OOB error
estimate.

\subsubsection{SVM}\label{svm}

Support vector machines(SVM) is the theory based on statistical learning
theory. It realizes the theory of VC dimension and principle of
structural risk minimum(SRM). The idea of SVM is to search an optimal
hyper-plane :TODO:

Suppose we are given a set of training data $xi \in R^n(i=1,2,…,n)$ with
the desired output $yi∈{+1,-1}$ corresponding to the two classes. And
suppose the dataset is linear seperable. So there exists a separating
hyper plane with the target functions w·xi+b=0 (w represents the weight
vector and b represents the bias). To ensure that all training data can
be classified, we must make the margin of separation $(2/‖w‖)$ maximum.
Then, in the case of linear separation, the linear SVM for optimal
separating hyper plane has the following optimization problem.

\[
\begin{aligned}\max & \frac{2}{||w||}\\
\text{s.t.} & y^{(i)}(w^{T}x^{(i)}+b)\geq1,i=1,\cdots,n\\
\end{aligned}
\]

the model above can be transformed as:

\[
\begin{aligned}\min & \frac{1}{2}||w||^{2}\\
\text{s.t.} & y^{(i)}(w^{T}x^{(i)}+b)\geq1,i=1,\cdots,n
\end{aligned}
\]

\subsection{Results and Discussion}\label{results-and-discussion}

\subsubsection{The data}\label{the-data}

Our aim is to predict whether a company would experience financial
failure in two years after based on the financial statement data of
current year. The data are from XXXX. We collected data from 2008 to
2013 but only use 2008-2010 for prediction. for example, we obtain data
of a company in its 2010 financial statement and this company was
classifed as special treatment(ST*) in 2013 for the first time. we pair
the data in 2010 and the label ST in 2013 together to build our model.
Once the model was built, we can predict wether a specific company will
be classified as special treatment two years after.

\subsubsection{procedure}\label{procedure}

First, we calculate DEA efficiency of the corperations in each year,an d
use the efficiency as a feature. Second, we caluclate viarable
importance through random forests. third, we build our models using the
3,5,10 most important variables. When building our models, we randmonly
select 200 nonST corperatons and resample ST corperations to 200 to make
two class balance. forth, we randomly divide the 400 sample into
training set and testing set. fifth, we use the training set to train
different models. sixth, we compared these models.

\subsubsection{results}\label{results}

The results shows that efficiency is an relatively important factor in
predicting ST

\begin{verbatim}
## 
## Recursive feature selection
## 
## Outer resampling method: Bootstrapped (25 reps) 
## 
## Resampling performance over subset size:
## 
##  Variables Accuracy Kappa AccuracySD KappaSD Selected
##          2    0.962 0.924    0.01074  0.0214         
##          3    0.965 0.930    0.00761  0.0152         
##          4    0.960 0.920    0.01056  0.0210         
##          5    0.963 0.925    0.01088  0.0217         
##          6    0.964 0.928    0.01168  0.0233         
##          7    0.966 0.932    0.00996  0.0199         
##          8    0.968 0.937    0.01052  0.0210         
##          9    0.967 0.935    0.01022  0.0204         
##         10    0.969 0.939    0.00964  0.0193         
##         11    0.970 0.939    0.01008  0.0201         
##         12    0.970 0.941    0.00960  0.0192         
##         13    0.970 0.940    0.00955  0.0191         
##         14    0.970 0.941    0.01049  0.0209         
##         15    0.971 0.942    0.00974  0.0194         
##         16    0.970 0.941    0.00951  0.0190         
##         17    0.971 0.943    0.00905  0.0181         
##         18    0.971 0.941    0.01007  0.0201         
##         19    0.972 0.943    0.00937  0.0187         
##         20    0.972 0.944    0.00944  0.0188         
##         21    0.972 0.945    0.00988  0.0197         
##         22    0.974 0.947    0.00957  0.0191         
##         23    0.974 0.948    0.00975  0.0195         
##         24    0.975 0.951    0.00839  0.0167         
##         25    0.975 0.951    0.00747  0.0149         
##         26    0.975 0.951    0.00767  0.0153         
##         27    0.976 0.952    0.00754  0.0151        *
## 
## The top 5 variables (out of 27):
##    T60800, T70600, T40402, T10300, T80403
\end{verbatim}

It is shown that every variable do contribute to the predition accuracy,
and DEA efficiency rank the 6th. for simplicity, we choose the first 3,
5, 10 most important variables, explore the difference of accuracy using
a variaty of machine learning methods.

\begin{verbatim}
##                    accuracy_without_eff accuracy_with_eff
## randomForest3vars                0.9750            0.9750
## randomForest5vars                0.9708            0.9667
## randomForest10vars               0.9708            0.9750
## SVM3vars                         0.9917            0.9875
## SVM5vars                         1.0000            1.0000
## SVM10vars                        1.0000            1.0000
## NaiveBayes3vars                  0.8208            0.8208
## NaiveBayes5vars                  0.8667            0.8375
## NaiveBayes10vars                 0.8542            0.8625
\end{verbatim}

\begin{Shaded}
\begin{Highlighting}[]
\CommentTok{# require(caret)}
\CommentTok{# require(randomForest)}
\CommentTok{# fit_rf = randomForest(Would_be_ST~.,data = training,importance = T)}
\CommentTok{# pred_rf = predict(fit_rf,newdata = testing)}
\CommentTok{# confusionMatrix(pred_rf,testing$Would_be_ST)}
\CommentTok{# varImpPlot(fit_rf)}
\CommentTok{# important_vars = importance(fit_rf,type = 1)}
\CommentTok{# }
\end{Highlighting}
\end{Shaded}

From the figure XX,we can see that the importance of variable
\textbf{efficiency} is not that great.

\begin{Shaded}
\begin{Highlighting}[]
\CommentTok{# library(RoughSets)}
\CommentTok{# decision_table = SF.asDecisionTable(mysample[,-c(1:12)],decision.attr=22,indx.nominal=22)}
\CommentTok{# reduct_quick = FS.greedy.heuristic.reduct.RST(decision_table,decisionIdx=22,qualityF=X.entropy)}
\CommentTok{# newDecTable2 = SF.applyDecTable(decision_table,reduct_quick)}
\end{Highlighting}
\end{Shaded}

\subsection{Conclusion}\label{conclusion}

\subsection{Reference}\label{reference}

\begin{enumerate}
\def\labelenumi{\arabic{enumi}.}
\item
  Breiman, Leo (2001). ``Random Forests''. Machine Learning 45 (1):
  5--32. \url{doi:10.1023/A:1010933404324}.
\item
\end{enumerate}

\end{document}
